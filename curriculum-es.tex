%% start of file `template.tex'.
%% Copyright 2006-2010 Xavier Danaux (xdanaux@gmail.com).
%
% This work may be distributed and/or modified under the
% conditions of the LaTeX Project Public License version 1.3c,
% available at http://www.latex-project.org/lppl/.

\documentclass[11pt,a4paper,sans]{moderncv}

% moderncv themes
\moderncvstyle{casual}                        % style options are 'casual' (default), 'classic', 'oldstyle' and 'banking'
\moderncvicons{awesome}


\moderncvcolor{green}                          % color options 'blue' (default), 'orange', 'green', 'red', 'purple', 'grey' and 'black'

%\ifpdf
%\RequirePackage[pdftex,]{hyperref}
%\else
%\RequirePackage[dvips]{hyperref}
%\fi

\usepackage[unicode,unicode,pdfpagelabels=false]{hyperref}

% character encoding
\usepackage[utf8]{inputenc}                   % replace by the encoding you are using
% Problema con el castellano
%\usepackage[spanish,es-noshorthands]{babel} es-notilde
%\usepackage[spanish]{babel}

% adjust the page margins
\usepackage{textcomp}

\usepackage[scale=0.8]{geometry}
%\setlength{\hintscolumnwidth}{3cm}						% if you want to change the width of the column with the dates
%\AtBeginDocument{\setlength{\maketitlenamewidth}{6cm}}  % only for the classic theme, if you want to change the width of your name placeholder (to leave more space for your address details
%\AtBeginDocument{\recomputelengths}                     % required when changes are made to page layout lengths

% personal data
\firstname{Arturo}
\familyname{Candela Moltó}
\title{Currículum Vítae}               % optional, remove the line if not wanted
\address{Alfonso El Magnánimo 13}{Alcoy, Alicante}    % optional, remove the line if not wanted
\mobile{+34618037813}                    % optional, remove the line if not wanted
%\phone{phone (optional)}                      % optional, remove the line if not wanted
%\fax{fax (optional)}                          % optional, remove the line if not wanted
\email{info@arturocandela.net}                      % optional, remove the line if not wanted
\homepage{arturocandela.net}                % optional, remove the line if not wanted
\extrainfo{Nacimiento - marzo 20, 1984}

%\extrainfo{additional information (optional)} % optional, remove the line if not wanted
\photo[64pt][0pt]{yo_pro_redondo}                         % '64pt' is the height the picture must be resized to and 'picture' is the name of the picture file; optional, remove the line if not wanted
%\quote{El emprendedor no es un soñador, es un hacedor}                
%\quote{Todo viaje de mil millas empieza con el primer paso, Lao-tzu}
% optional, remove the line if not wanted

% to show numerical labels in the bibliography; only useful if you make citations in your resume
\makeatletter
\renewcommand*{\bibliographyitemlabel}{\@biblabel{\arabic{enumiv}}}
\makeatother

% bibliography with mutiple entries
\usepackage{multibib}
%\newcites{book,misc}{{Books},{Others}}
\newcites{book,web,android,ios}{{Publicaciones},{Webs},{Aplicaciones Android},{Aplicaciones iOS}}

\hypersetup{
    colorlinks=true,
       urlcolor=color1,
       linkcolor=color1,
       pdfpagelayout=SinglePage,
       pdfnewwindow=true,
       pdfstartview=FitV
}

\nopagenumbers{}                             % uncomment to suppress automatic page numbering for CVs longer than one page
%----------------------------------------------------------------------------------
%            content
%----------------------------------------------------------------------------------
\begin{document}


%\clearpage
%-----       letter       ---------------------------------------------------------
% recipient data
\recipient{Recursos Humanos}{From the Bench S.L.\\Plaza Mayor 9 \\Elda}
\date{Noviembre 22, 2014}
\opening{Estimados Señores,}
\closing{Atentamente,}
\enclosure[Adjunto]{curriculum vit\ae{}}          % use an optional argument to use a string other than "Enclosure", or redefine \enclname
\makelettertitle

Me dirijo a ustedes en respuesta a la oferta de empleo publicada en su web para el puesto de \textbf{Desarrollador de Android}.  


Soy Máster en Software Libre e Ingeniero Técnico en Informática de Gestión y Telecomunicaciones en la especialidad de Telemática. Llevo trabajando 8 años en el sector de las las TICs. Sobre todo, durante los últimos 5 años ha sido cuando mi trabajo se ha centrado en el desarrollo. Durante mi época como Investigador en la Universidad Politécnica fui responsable del diseño y posterior programación de los proyectos de los que formé parte, la mayoría basados en computación ubicua y entornos de cliente servidor utilizando bases de datos NoSQL. Además durante esa época trabajé como profesional por cuenta propia realizando otros trabajos como desarrollador para iOS y Android, tal como podrán observar en la sección del portfolio. Actualmente trabajo como ingeniero de calidad automatizando tareas y ayudando a los equipos directivos en el diseño de los Planes de Pruebas.

Les adjunto mi CV porque estoy interesado en continuar el desarrollo de mi carrera profesional su empresa \textbf{From the Bench, S.L.} participando como programador en sus aplicaciones que actualmente son utilizadas por millones de usuarios en todo el mundo.

\makeletterclosing
%\clearpage

\maketitle

\section{Experiencia}
\cventry{2014 -*}{Ingenierio de QA Software}{Embarcadero Technologies, Inc.}{Elche}{}{Automatización de las tareas de testeo y diseño de planes de pruebas para productos.}
\cventry{2010 -- 2014}{Técnico de Investigación y Desarrollo de Software}{Universidad Politécnica de Valencia}{Alcoy}{}{Desempeño de tareas de investigación en los proyectos de investigación TacTic y Prometeo. Investigación y desarrollo de algoritmos para renderizar información vectorial sobre entornos 2D y 3D para hacer esa información accesible en tiempo real. Desarrollo de aplicaciones para demostrar el funcionamiento de esos algoritmos.\newline{}\newline{}
Detalle de las aplicaciones/algoritmos desarrollados:
\begin{itemize}
	\item \textbf{Android}:
		\begin{itemize}
			\item Desarrollo de un nuevo algotitmo de renderizado 2D de terreno utilizando OpenGL ES para Android para mostrar información vectorial interactiva en tiempo real.
			\item Aplicación para administrar un dispositivo bluetooth para la monitorización de la tempratura en entornos médicos e industriales
			\item Aplicaciones para tener las últimas noticias de lo que está sucediendo en la ciudad que incluye información de: tiendas, eventos, teléfonos de interés, puntos de interés en un mapa, rutas cicloturistas. 
		\end{itemize}
	\item \textbf{iOS}:
		\begin{itemize}
			\item Aplicación para un museo que tiene un juego y un lector de códigos Bidi (QR) con el que el usuario puede encontrar tesoros escondidos y ayudarle si se pierde.
			\item Aplicaciones interactivas para estar al tanto de las últimas noticias de lo que está sucediendo en la ciudad y que incluye información de: tiendas, eventos, teléfonos de interés, puntos de interés en un mapa, rutas cicloturistas. 
		\end{itemize}
	\item \textbf{nodejs}
		\begin{itemize}
			\item Aplicación de backend que provee de WebServices utilizando el estándar GeoJSON para la aplicación de Android de renderizado de terreno. Utiliza PostgreSQL con el complemento PostGIS y la librería sockets.io para proveer de comunicación en tiempo real.
			\item Aplicación con JQueryUI para el front-end y para el backend se utilizó nodejs con la base de datos NoSQL CouchDB (couchbase) con la extensión GeoCouch para realizar una aplicación web de edición de información geolozalizada en un mapa en tiempo real.
		\end{itemize}
\end{itemize}}

\cventry{2009}{Técnico de Investigación y Desarrollo de Software}{Instituto Tecnológico de Informática}{Alcoy}{}{Responsable de desarrollo del proyecto VirtualCar para la visualización de automóviles modelizados.}
\cventry{2006 -- 2008}{Responsable de Informática}{Cámara Oficial de Comercio e Industria de Alcoy}{}{}{Gestión de la red interna, gestión de las bases de datos y mantenimiento de la web. }%\newline{} Responsable del programa Nexopyme.

\section{Formación Reglada}
\cventry{2006--2008}{Máster Universitario en Software Libre}{Universitat Oberta de Catalunya}{Barcelona}{Nota media: 8,20}{}  % arguments 3 to 6 can be left empty
\cventry{2006--2009}{Ingeniero Técnico en Informática de Gestión}{Universidad Politécnica de Valencia}{Alcoy}{Nota media: 7,1}{}
\cventry{2002--2006}{Ingeniero Técnico de Telecomunicaciones especialidad Telemática}{Universidad Politécnica de Valencia}{Alcoy}{ Nota media 6,7}{}
\cventry{2009}{Certificado de Aptitud Pedagógica}{Instituto de ciencias de la Educación de la UPV}{Alcoy}{}{}
%\cventry{2006}{Grau Mitjà de Coneixements de Valencià}{Junta Qualificadora de Coneixements de Valencià}{}{}{}

\section{Formación Complementaria}

\cventry{2013}{Principios de programación funcional en Scala}{École Polytechnic Fédérale de Lausanne}{}{}{}

\cventry{2012}{Brainstorm Users Meeting de 40 horas}{Brainstorm Multimedia S.L}{}{}{}

\cventry{2012}{Curso de Programación para Android de 250 horas}{Escuela Universitaria de
Formación Abierta del Grupo Master.D}{Universidad Camilo José Cela}{}{}

\cventry{2011}{Desarrollo de Juegos y Aplicaciones Interactivas Multiplataforma (PC, MAC, WEB, IOS, ANDROID) con Unity3D de 40 horas}{Curso de Formación Permanente}{UPV}{}{}
\cventry{2009}{Documentos Técnicos y Científicos con \LaTeX de 40 horas}{Curso de Formación Permanente}{UPV}{}{}

\cventry{2009}{Introducción a la programación en C\# y .NET en Windows y Linux de 40 horas}{Curso de Formación Permanente}{UPV}{}{}

\cventry{2008}{Desarrollo Web avanzado mediante AJAX/GWT (Google Web Toolkit) de 20 horas}{Curso de Formación Permanente}{UPV}{}{}
\cventry{2006}{Servidores Web de 20 horas}{Curso de Formación de Permanente}{UPV}{}{Incluye configuración del servidor web Apache y de Zope.}
%\cventry{2006}{Autowebmarketing. Autogestión del Sitio Web Corporativo como Soporte del Plan de Marketing Empresarial de 12 horas}{Academia de las Cámaras}{Consejo Superior de Cámaras de Comercio}{Madrid}{}
%\cventry{2005}{Seguridad en Redes Corporativas de 30 horas}{Curso de Formación Permanente}{UPV}{}{}
%\cventry{2004}{Montaje, Planificación y Configuración de Redes Inalámbricas 20 horas}{Curso de Formación Permanete}{UPV}{}{}


%Formacion Antigua
%\cventry{2000}{Curso de Windows'95 y de Técnico de Programación en Turbo Pascal 140 horas}{ Centro de Formación – M@rqus}{Alcoy}{}{}
%\cventry{1999}{Curso "Introducción al montaje y reparación del PC"}{Centro de Formación – M@rqus}{Alcoy}{}{}

\section{Idiomas}
\cvitemwithcomment{Valenciano}{Nivell Mitjà}{Certificado por la "Junta Qualificadora de Coneixments de Valencià" en 2006}
\cvitemwithcomment{Castellano}{Nivel Alto}{Nativo}
\cvitemwithcomment{Inglés}{Nivel Medio}{}

\section{Informática}
\cvdoubleitem{Lenguajes}{objective-C, Java, JavaScript, Python, C, C++, \href{http://www.php.net/}{PHP}, C\#, Pascal}{Plataformas Móviles}{\href{http://developer.android.com/}{Android},\href{http://www.apple.com/es/ios/}{iOS} (iPhone e iPad)}
\cvdoubleitem{Bases de Datos}{\href{http://www.mysql.com/}{MySQL}, \href{http://www.postgresql.org/}{PostgreSQL},\href{http://couchdb.apache.org/}{couchdb},
\href{http://www.sqlite.org/}{SQLite}}{Desarrollo en Grupo}{\href{http://trac.edgewall.org/}{Trac}, git, sourcetree,\href{https://es.atlassian.com/software/jira/}{Jira}, \href{http://subversion.apache.org/}{subversion},
\href{http://tortoisesvn.tigris.org/}{TortoiseSVN}}
\cvdoubleitem{Entornos de Desarrollo}{\href{https://developer.apple.com/xcode/}{Xcode}, \href{http://eclipse.org/}{Eclipse}, \href{http://msdn.microsoft.com/es-es/vstudio/default}{Visual Studio} }{}{}



%\section{Extra 1}
%\cvlistitem{Item 1}
%\cvlistitem{Item 2}
%\cvlistitem[+]{Item 3}            % optional other symbol
%
%\renewcommand{\listitemsymbol}{-} % change the symbol for lists
%
%\section{Extra 2}
%\cvlistdoubleitem{Item 1}{Item 4}
%\cvlistdoubleitem{Item 2}{Item 5}
%\cvlistdoubleitem{Item 3}{ }

% Publications from a BibTeX file without multibib\renewcommand*{\bibliographyitemlabel}{\@biblabel{\arabic{enumiv}}}% for BibTeX numerical labels
%\nocite{*}
%\bibliographystyle{plain}
%\bibliography{publications}       % 'publications' is the name of a BibTeX file

% Publications from a BibTeX file using the multibib package
\section{Portfolio}

%\nociteandroid{android:mislugares}
\nociteandroid{android:wheelingmobileapp}
\nociteandroid{android:thercomapp}
\nociteandroid{android:elhondoapp}
\nociteandroid{android:catralcityapp}

\bibliographystyleandroid{plain}
\bibliographyandroid{publications}

\nociteios{ios:westvirginiamuseumapp}
\nociteios{ios:catralcityapp}
\nociteios{ios:elhondocityapp}
\nociteios{ios:oglebayapp}

\bibliographystyleios{plain}
\bibliographyios{publications}

\nocitebook{graficos_max_potencia}
\nocitebook{mobweb}
\bibliographystylebook{plain}
\bibliographybook{publications}   % 'publications' is the name of a BibTeX file

\section{Otros}
\cvlistitem{Permiso de conducir B}

\end{document}


%% end of file `template_en.tex'.
