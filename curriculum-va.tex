%% start of file `template.tex'.
%% Copyright 2006-2010 Xavier Danaux (xdanaux@gmail.com).
%
% This work may be distributed and/or modified under the
% conditions of the LaTeX Project Public License version 1.3c,
% available at http://www.latex-project.org/lppl/.

\documentclass[11pt,a4paper,sans]{moderncv}
%\usepackage[firstyear=2006,lastyear=2015]{moderntimeline}
%\tlwidth{2.0ex}


% moderncv themes
\moderncvstyle{casual}                        % style options are 'casual' (default), 'classic', 'oldstyle' and 'banking'
\moderncvicons{awesome}


\moderncvcolor{green}                          % color options 'blue' (default), 'orange', 'green', 'red', 'purple', 'grey' and 'black'

%\ifpdf
%\RequirePackage[pdftex,]{hyperref}
%\else
%\RequirePackage[dvips]{hyperref}
%\fi

\usepackage[unicode,unicode,pdfpagelabels=false]{hyperref}

% character encoding
\usepackage[utf8]{inputenc}                   % replace by the encoding you are using
% Problema con el castellano
%\usepackage[spanish,es-noshorthands]{babel} es-notilde
%\usepackage[spanish]{babel}
\usepackage[catalan]{babel}


% adjust the page margins
\usepackage{textcomp}

%\usepackage[document]{ragged2e}

\usepackage[scale=0.8]{geometry}
\setlength{\hintscolumnwidth}{2.5cm}          % si desea cambiar el ando de la columna para las fechas
%\setlength{\hintscolumnwidth}{3cm}						% if you want to change the width of the column with the dates
%\AtBeginDocument{\setlength{\maketitlenamewidth}{6cm}}  % only for the classic theme, if you want to change the width of your name placeholder (to leave more space for your address details
\AtBeginDocument{\recomputelengths}                     % required when changes are made to page layout lengths

% personal data
\firstname{Arturo}
\familyname{Candela i Moltó}
\title{Currículum Vit\ae}               % optional, remove the line if not wanted
\address{Alfons El Magnànim 13}{Alcoi, Alacant}    % optional, remove the line if not wanted
\mobile{+34618037813}                    % optional, remove the line if not wanted
%\phone{phone (optional)}                      % optional, remove the line if not wanted
%\fax{fax (optional)}                          % optional, remove the line if not wanted
\email{info@arturocandela.net}                      % optional, remove the line if not wanted
\homepage{arturocandela.net}                % optional, remove the line if not wanted
\extrainfo{Naixement - març 20, 1984}

%\extrainfo{additional information (optional)} % optional, remove the line if not wanted
\photo[64pt][0pt]{yo_pro_redondo}                         % '64pt' is the height the picture must be resized to and 'picture' is the name of the picture file; optional, remove the line if not wanted
%\quote{El emprendedor no es un soñador, es un hacedor}                
%\quote{Todo viaje de mil millas empieza con el primer paso, Lao-tzu}
% optional, remove the line if not wanted

% to show numerical labels in the bibliography; only useful if you make citations in your resume
\makeatletter
\renewcommand*{\bibliographyitemlabel}{\@biblabel{\arabic{enumiv}}}


\makeatother

% bibliography with mutiple entries
\usepackage{multibib}
%\newcites{book,misc}{{Books},{Others}}
\newcites{book,web,android,ios}{{Publicacions},{Webs},{Aplicacions Android},{Aplicacions iOS}}

\hypersetup{
    colorlinks=true,
       urlcolor=color1,
       linkcolor=color1,
       pdfpagelayout=SinglePage,
       pdfnewwindow=true,
       pdfstartview=FitV
}

\nopagenumbers{}                             % uncomment to suppress automatic page numbering for CVs longer than one page
%----------------------------------------------------------------------------------
%            content
%----------------------------------------------------------------------------------
\begin{document}

%%%%%%%%%%%%%%%%%%%%%%%%%%%%%%%%%%%%%%%%%%%%%%%%%%%%%%%%%%%%%
%
%  Carta de presentació
%
%%%%%%%%%%%%%%%%%%%%%%%%%%%%%%%%%%%%%%%%%%%%%%%%%%%%%%%%%%%%%%

%%%%%%%%%%%%%%%%%%%%%%%%%%%%%%%%%%%%%%%%%%%%%%%%%%%%%%%%%%%%%
%
%  Carta de presentació
%
%%%%%%%%%%%%%%%%%%%%%%%%%%%%%%%%%%%%%%%%%%%%%%%%%%%%%%%%%%%%%%


\recipient{Escola de Doctorat}{Av. Carl Friedrich Gauss, 5\\08860 Castelldefels\\UOC - Castelldefels PMT}
\date{\today}
\opening{Benvolguts membres de la Comisió Acadèmica del programa de Doctorat en Tecnologies de la Informació i Xarxes,}
\closing{Atentament,}
\enclosure[Adjunt]{curriculum vit\ae{}}          % use an optional argument to use a string other than "Enclosure", or redefine \enclname




\makelettertitle

Després d'haver finalitzat els meus estudis de \textbf{Màster en Programari Lliure} de la UOC i haver treballat durant X anys en Investigació, us envie la meva sol·licitud i candidatura per al Programa de Doctorat en Tecnologies de la Informació i Xarxes que la Universitat Oberta de Catalunya ofereix.

Els meus estudis i treball han anat sempre al voltant del desenvolupament de programari. Als meus estudis, les matèries que més m'han motivat han segut totes les relacionades en programació, xarxes de computadors, estructures de dades i algoritmes, disseny de bases de dades i seguretat informàtica. Durant el meu treball en el camp de de la investigació, he dissenyat i implementat tota mena de sistemes informàtics: sistemes 3D, de bases de dades distribuïdes i geogràfiques, aplicacions per a mòbils en diverses plataformes, etcètera.

Particularment estic interessat en el programa de \textbf{Doctorat en Tecnologies de la Informació i Xarxes} en la línia de recerca \textbf{Design of e-healthcare systems as a step forward towards their improvement and sustainability} perquè vull aprofundir i dotar de sentit al coneixements i habilitats que he anat adquirint durant els anys de manera que puga retornar a la societat una part del que ha invertit en mi durant els anys.

Voldria realitzar el programa de Doctorat, i en concret a la UOC, perquè ja tinc l'experiència d'haver realitzat el \textit{Màster Universitari en Programari Lliure} que fou molt gratificant. Posteriorment a la UOC he cursat diversa formació on-line no universitària i la veritat es que no he quedat ni de lluny tan satisfet com amb vosaltres: el materials estaven des-actualitzats, el seguiment dels professors no era tan exhaustiu com el vostre i a mes, vosaltres doneu un servei d'atenció molt orientat a la persona; encara que la formació és a distancia es quasi com si us tingues al costat. 


\makeletterclosing

\clearpage


%\clearpage
%-----       letter       ---------------------------------------------------------
% recipient data
\recipient{Recursos Humanos}{From the Bench S.L.\\Plaza Mayor 9 \\Elda}
\date{Noviembre 22, 2014}
\opening{Estimados Señores,}
\closing{Atentamente,}
\enclosure[Adjunto]{curriculum vit\ae{}}          % use an optional argument to use a string other than "Enclosure", or redefine \enclname
\makelettertitle

Me dirijo a ustedes en respuesta a la oferta de empleo publicada en su web para el puesto de \textbf{Desarrollador de Android}.  


Soy Máster en Software Libre e Ingeniero Técnico en Informática de Gestión y Telecomunicaciones en la especialidad de Telemática. Llevo trabajando 8 años en el sector de las las TICs. Sobre todo, durante los últimos 5 años ha sido cuando mi trabajo se ha centrado en el desarrollo. Durante mi época como Investigador en la Universidad Politécnica fui responsable del diseño y posterior programación de los proyectos de los que formé parte, la mayoría basados en computación ubicua y entornos de cliente servidor utilizando bases de datos NoSQL. Además durante esa época trabajé como profesional por cuenta propia realizando otros trabajos como desarrollador para iOS y Android, tal como podrán observar en la sección del portfolio. Actualmente trabajo como ingeniero de calidad automatizando tareas y ayudando a los equipos directivos en el diseño de los Planes de Pruebas.

Les adjunto mi CV porque estoy interesado en continuar el desarrollo de mi carrera profesional su empresa \textbf{From the Bench, S.L.} participando como programador en sus aplicaciones que actualmente son utilizadas por millones de usuarios en todo el mundo.

\makeletterclosing
%\clearpage

\maketitle

%%%%%%%%%%%%%%%%%%%%%%%%%%%%%%%%%%%%%%%%%%%%%%%%%%%%%%%%%%%%%
%
%  Experiencia
%
%%%%%%%%%%%%%%%%%%%%%%%%%%%%%%%%%%%%%%%%%%%%%%%%%%%%%%%%%%%%%%

\section{Experiència}

%UPV
\cventry{2016 -- *}{Tècnic d'investigació i Desenvolupador de Software}{Universitat Politècnica de València}{Alcoi}{}{
Desenvolupament de tasques d'investigació al projectes Ontime, \href{https://www.eurostars-eureka.eu/project/id/9831}{HAI-OPS:}\textit{ Hospital Acquired Infection and Outbreak Prevention System}, \href{https://www.eurostars-eureka.eu/project/id/9770}{i-Light:}\textit{A pervasive home monitoring system based on intelligent luminaires}, diversos projectes conjuntament amb l'Hospital Universitari Politècnic La Fe i altres tasques:\newline{}
\begin{itemize}
	\item \textbf{Ontime:}
		\begin{itemize}
			\item Disseny i implementació de l'estructura de dades per a donar suport a un sistema informàtic per al suport de dels transports en \textit{HEMS - Helicòpters dels Serveis d'Emergències Mèdiques}.
			\item Disseny de l'estratègia per a integrar diferents tipus de sensors metges, ambientals i òptics a un Framework comú y posterior integració.
			\item Desenvolupament d'una aplicació Android de prova de concepte del sistema HEMS anterior per a donar suport a la feina del metge que realitza el transport
		\end{itemize}
	\item \textbf{HAI-OPS and i-Light} 
	\begin{itemize}
		\item Disseny del protocol de comunicacions JSON/REST per que els diversos  dispositius pogueren comunicar-se, la base de dades i una base de dades per a poder desar tota la informació
		\end{itemize}		
\end{itemize}
}

%Everis
\cventry{2015--2016}{Analista de Software Sènior en Mobilitat}{Everis Centers, S.L.}{Alacant}{}{Desenvolupament, disseny i implementació de solucions per a la plataforma mòbil iOS. Formació dels companys en les tecnologies utilitzades. Tasques desenvolupades:\newline{}
\begin{itemize}
	\item \textbf{Git:} Concepció de l'estratègia per a combinar els diversos desenvolupaments de l'equip amb submòduls i diverses rames
	\item \textbf{Jenkins:} Instal·lació i configuració del codi font per a realitzar testing automatitzat sobre el codi també per a que realitzara diversos anàlisis estàtics
	\item \textbf{SonarQube:} Instal·lació i configuració d'aquesta ferramenta per a obtenir mes tipus d'anàlisi estàtic de codi.
\end{itemize}
}

%Embarcadero
\cventry{2014--2015}{Enginyer QA Software}{Embarcadero Technologies, Inc.}{Elx}{}{Automatització de les tasques de testeig i disseny de plans de proves per al producte: \textit{ER/Studio Data Architect}\newline{}
\begin{itemize}
	\item Integració d'un dels components de la interfície gràfica amb la interfície IAccessible de \textit{Microsoft} per a que fora visible mitjançant la ferramenta CodedUI per a poder testejar
	\item Muntatge de tota la infraestructura necessària per poder validar que \textit{ER/Studio Data Architect}  podia integrar-se amb \textbf{MongoDB}
	\item Escriptura de tests generals del producte i a més per a tot el procés de certificació de \textbf{MongoDB} 
	\item Automatització de tests de la interfície gràfica utilitzant CodedUI
\end{itemize}
}

%UPV
\cventry{2010--2014}{Tècnic d'investigació i Desenvolupador de Software}{Universitat Politècnica de València}{Alcoi}{}{
Desenvolupament de tasques d'investigació en el projectes d'investigació Avanza TacTic financiat pel Ministeri d'Indústria i Comerç i el projecte CENIT Prometeo per a la lluita contra incendis financiat per Centre de Desenvolupament Tecnològic Industria. Investigació i desenvolupament d'algoritmes per a renderitzar informació vectorial en entorns 3D i 2D i fer-la accessible en temps real. Desenvolupament d'aplicacions per a demostrar el seu funcionament.Detall de les aplicacions desenvolupades:\newline{}
\begin{itemize}
	\item \textbf{Android}:
		\begin{itemize}
			\item Desenvolupament d'un nou algoritme de renderitzat 2D de terreny per Android basat en OpenGL ES per a mostrar informació vectorial interactiva en temps reial
			\item Aplicació per a administrar el dispositiu per al mesurament i control de temperatura en entorns metges e industrials \httplink[Thercom]{http://innovatecsc.com/es/portfolio_category/thercom/} 
			\item Desenvolupament d'aplicacions \textit{SmartCities} i de tota la seva infraestructura de servidor per a mostrar i actualitzar informació de: tendes, events, telèfons d'interès, punts d'interès a un mapa, rutes cicloturistes
		\end{itemize}
	\item \textbf{iOS}:
		\begin{itemize}
			\item Aplicacio per a un museu que inclou un joc i un lector de codis Bidi(QR) de manera que l'usuari pot trobar tresors amagats i obtindre la localització a un mapa si es perd. 
			\item Desenvolupament d'aplicacions \textit{SmartCities} i de tota la seva infraestructura de servidor per a mostrar i actualitzar informació de: tendes, events, telèfons d'interès, punts d'interès a un mapa, rutes cicloturistes
		\end{itemize}
	\item \textbf{nodejs}:
		\begin{itemize}
		 \item Aplicacio de \textit{backend} que que proveïx de WebServices utilitzant l'estàndard GeoJSON per a una aplicació d'android de renderizat de terreny. Utilitza PostgreSQL amb el complement PostGIS i la llibreria Sockets.io per a dotar el sistema de comunicacio en temps real. A part de l'aplicacio d'Android, el sistema permetia diversos clients, i a mes es va connectar al Middleware ZeroC Ice per a que s'interconnectara amb diverses fonts.
		 \item Aplicacio JQueryUI per al \textit{frontend} i per al backend nodejs amb la base de dades NoSQL CouchDB (couchbase) amb el complement GeoCouch per a realitzar una aplicacio web d'edició d'informació geolocalitzada en un mapa en temps real  
		\end{itemize}
\end{itemize}
}

%ITI
\cventry{2009}{Tècnic d'investigació i Desenvolupador de Software}{Institut Tecnològic d'Informàtica}{Alcoi}{}{Responsable del desenvolupament i el disseny del projecte VirtualCar per a la visualització d'automòbils modelitzats per computador. Tasques Desenvolupades:\newline{}
\begin{itemize}
	\item Procés de totes les parts del cotxe i creació de peces en diferents nivell de detall per a que el motor gràfic funcionara en temps real, per a que segons com de prop estiga de l'usuari mostre unes amb mes resolució o altres amb menys.
	\item Creació d'aplicacions de manera que els fabricants pugen mantenir la informació que el VirtualCar mostra d'ells actualitzada.
	\item Realització de totes les funcions necessàries per a dotar d'interactivitat al producte
\end{itemize}
}

%Cámara
\cventry{2006--2008}{Responsable d'Informàtica}{Cambra Oficial de Comerç i Industria d'Alcoi}{Alcoi}{}{Gestió de la xarxa interna, de les bases de dades i manteniment de la Web. Tasques Desenvolupades:\newline{}
\begin{itemize}
	\item Responsable del programa NexoPyme - Fou un programa que pretenia contribuir a la millora de la competitivitat de les xicotets i mitjanes empreses ajudant-les a introduir-se en món on-line amb una serie d'actuacions molt concretes.
	\item Generació d'un programa de Quadre de Comandaments am l'aplicació QlikView que integrava informació de diverses bases de dades per a que el membres de la Cambra pogueren analitzar millor les dades. 
	\item Desenvolupament d'un software de gestió de xarxa que permet tenir unificades totes les dades en un mateix registre
\end{itemize}
}

\

%%%%%%%%%%%%%%%%%%%%%%%%%%%%%%%%%%%%%%%%%%%%%%%%%%%%%%%%%%%%%%
%
%  Formación Reglada
%
%%%%%%%%%%%%%%%%%%%%%%%%%%%%%%%%%%%%%%%%%%%%%%%%%%%%%%%%%%%%%%


\section{Formació Reglada}

\cventry{2009}{Certificat d'Aptitud Pedagògica}{Institut de Ciències de la Educació de la UPV}{Alcoi}{}{}

\cventry{2006--2008}{Màster Universitari en Programari Lliure}{Universitat Oberta de Catalunya}{Barcelona}{Nota mitjana en base 10: 8,20. En base 4: 2,21}{}  % arguments 3 to 6 can be left empty

\cventry{2006--2009}{Enginyer Tècnic en Informàtica de Gestió}{Universitat Politècnica de València}{Alcoi}{Nota mitjana en base 10: 7,1. En base 4: 1,69}{}

\cventry{2002--2006}{Enginyer Tècnic en Telecomunicacions, especialitat Telemàtica}{Universitat Politècnica de València}{Alcoy}{ Nota mitjana en base 10: 6,7. En base 4:1,59}{}

%%%%%%%%%%%%%%%%%%%%%%%%%%%%%%%%%%%%%%%%%%%%%%%%%%%%%%%%%%%%%%
%
%  Formación Complementaria
%
%%%%%%%%%%%%%%%%%%%%%%%%%%%%%%%%%%%%%%%%%%%%%%%%%%%%%%%%%%%%%%
\section{Cursos}

\cventry{2013}{Principis de la programació funcional en Scala}{Escola Politècnica Federal de Lausanne}{}{}{}

\cventry{2012}{Brainstorm Users Meeting de 40 hores}{Brainstorm Multimedia S.L}{}{}{}

\cventry{2012}{Curs de Programació per a Android de 250 hores}{Escola Universitària de formació oberta del Grup Master.D, Universidad Camilo José Cela}{Madrid}{}{}

\cventry{2011}{Desenvolupaments de Jocs i Aplicacions Interactives Multi-plataforma (PC, MAC, WEB, IOS, ANDROID) amb Unity3D de 40 hores}{Centre de Formació Permanent}{UPV}{}{}
\cventry{2009}{Documents Tècnics i Científics amb \LaTeX de 40 hores}{Centre de Formació Permanent de la UPV}{UPV}{}{}

\cventry{2009}{Introducció a la programació en C\# y .NET a Windows i Linux de 40 hores}{Centre de Formació Permanent de la UPV}{UPV}{}{}

\cventry{2008}{Desenvolupament Web Avançat mitjançant AJAX/GWT (Google Web Toolkit) de 20 hores}{Centre de Formació Permanent de la UPV}{UPV}{}{}

\cventry{2006}{Servidors Web de 20 hores}{Centre de Formació Permanent de la UPV}{UPV}{}{Inclou la configuració del servidor web Apache i de Zope}

%\cventry{2006}{Autowebmarketing. Autogestión del Sitio Web Corporativo como Soporte del Plan de Marketing Empresarial de 12 horas}{Academia de las Cámaras}{Consejo Superior de Cámaras de Comercio}{Madrid}{}

\cventry{2005}{Seguretat en Xarxes corporatives de 30 hores}{Centre de Formació Permanent de la UPV}{UPV}{}{}

\cventry{2004}{Planificació i Configuració de Xarxes Wi-Fi de 20 hores}{Centre de Formació Permanent de la UPV}{UPV}{}{}


%Formacion Antigua
\cventry{2000}{Tècnic de programació en Turbo Pascal de 140 hores}{ Centre de formació M@rqus}{Alcoi}{}{}
%\cventry{1999}{Introducción al montaje y reparación del PC}{Centro de Formación – M@rqus}{Alcoy}{}{}

%%%%%%%%%%%%%%%%%%%%%%%%%%%%%%%%%%%%%%%%%%%%%%%%%%%%%%%%%%%%%%
%
%  Idiomas
%
%%%%%%%%%%%%%%%%%%%%%%%%%%%%%%%%%%%%%%%%%%%%%%%%%%%%%%%%%%%%

\section{Idiomes}
\cvlanguage{Castellà}{Natiu}{}
\cvlanguage{Valencià}{C1}{\textbf{Certificat Mitjà} per la \textbf{Junta Qualificadora de Coneixements de Valencià},setembre de 2006}
\cvlanguage{Anglés}{B2}{\textbf{Cambridge English}: \textbf{First Certificate}, octubre de 2016}

\section{Informàtica}
\cvcomputer{Llenguatjes de Programació}{Objective-C, python, Java, C, C++, C\#}{Plataformes Mòbils}{Android, iOS (iPhone e iPad)}
\cvcomputer{Bases de Dades}{MongoDB, MySQL,PostgreSQL, SQLite, JavaDB,couchdb}{Ferramentes de Treball en grup}{JIRA, subversion, git (git-flow), SourceTree, Perforce, bitbucket, gitlab, gitbucket, trac}
\cvcomputer{IDE}{Android Studio, XCode, eclipse, Visual Studio, Netbeans}{Administració de sistemes}{Windows, Linux, MacOS}

%\section{Extra 1}
%\cvlistitem{Item 1}
%\cvlistitem{Item 2}
%\cvlistitem[+]{Item 3}            % optional other symbol
%
%\renewcommand{\listitemsymbol}{-} % change the symbol for lists
%
%\section{Extra 2}
%\cvlistdoubleitem{Item 1}{Item 4}
%\cvlistdoubleitem{Item 2}{Item 5}
%\cvlistdoubleitem{Item 3}{ }

% Publications from a BibTeX file without multibib\renewcommand*{\bibliographyitemlabel}{\@biblabel{\arabic{enumiv}}}% for BibTeX numerical labels
%\nocite{*}
%\bibliographystyle{plain}
%\bibliography{publications}       % 'publications' is the name of a BibTeX file

% Publications from a BibTeX file using the multibib package

\section{Dossier}

%\nociteandroid{android:mislugares}
\nociteandroid{android:wheelingmobileapp}
\nociteandroid{android:thercomapp}
\nociteandroid{android:elhondoapp}
\nociteandroid{android:catralcityapp}

\bibliographystyleandroid{plain}
\bibliographyandroid{publications}

\nociteios{ios:westvirginiamuseumapp}
\nociteios{ios:catralcityapp}
\nociteios{ios:elhondocityapp}
\nociteios{ios:oglebayapp}

\bibliographystyleios{plain}
\bibliographyios{publications}

\nocitebook{graficos_max_potencia}
\nocitebook{mobweb}
\bibliographystylebook{plain}
\bibliographybook{publications}   % 'publications' is the name of a BibTeX file


%%%%%%%%%%%%%%%%%%%%%%%%%%%%%%%%%%%%%%%%%%%%%%%%%%%%%%%%%%%%%%
%
%  Idiomas
%
%%%%%%%%%%%%%%%%%%%%%%%%%%%%%%%%%%%%%%%%%%%%%%%%%%%%%%%%%%%%


%\section{Altres}
%\cvlistitem{Permís de conduir B}
%\cvlistitem{Entusiasta del programari lliure}
%\cvlistitem{bricolador}
%\cvlistitem{brico-programador}
\end{document}


%% end of file `template_en.tex'.
